% !TeX spellcheck = lt
\documentclass{VUMIFPSbakalaurinis}
\usepackage{algorithmicx}
\usepackage{algorithm}
\usepackage{algpseudocode}
\usepackage{amsfonts}
\usepackage{amsmath}
\usepackage{bm}
\usepackage{caption}
\usepackage{color}
\usepackage{float}
\usepackage{graphicx}
\usepackage{listings}
\usepackage{subfig}
\usepackage{url}
\usepackage{wrapfig}
\usepackage[table,xcdraw]{xcolor}
\usepackage[backend=biber]{biblatex}
\usepackage{enumitem}\setlist{nosep}
\usepackage{csquotes}

% Titulinio aprašas
\university{Vilniaus universitetas}
\faculty{Informatikos institutas}
\department{Programų sistemos}
%\papertype{Bakalauro darbas}
\papertype{Bakalauro baigiamasis darbas}
\title{Sustiprinto mokymosi taikymas žaidimo agento valdymo programos kūrimui}
\titleineng{Application of reinforcement learning to the software development for game agent management}
\author{Jokūbas Rusakevičius}
\supervisor{vyresn. m.d. Virginijus Marcinkevičius}
\reviewer{}
\date{Vilnius – \the\year}

% Nustatymai
\setmainfont[ItalicFont 	= Palem3.2-it.ttf,
			BoldItalicFont	= Palem3.2-bi.ttf,
			BoldFont		= Palem3.2-bd.ttf]
			{Palem3.2-nm.ttf}
\bibliography{bibliografija}

\begin{document}
\maketitle

\setcounter{page}{2}

\sectionnonumnocontent{Santrauka}
TODO: Santrauka
% Nurodomi iki 5 svarbiausių temos raktinių žodžių (terminų).
% Vienas terminas gali susidėti iš kelių žodžių.
\raktiniaizodziai{, , , raktinis žodis 4, raktinis žodis 5}   

\sectionnonumnocontent{Summary}
TODO: summary
\keywords{, , , keyword 4, keyword 5}

\tableofcontents
z
\sectionnonum{Įvadas}

\section{Eksperimentas...}
Šiame skyriuje aprašomas bakalauro darbo metu atlikta praktinė dalis bei eksperimentinės aplinkos paruošimas.................

\subsection{Eksperimentinės aplinkos paruošimas}
Eksperimentas atliekamas naudojant \enquote{Jupyter Notebook}  ...............

\subsubsection{Eksperimentinė aplinka}
Eksperimentas atliekamas naudojantis realią \enquote{Windows 10} mašiną.

\begin{enumerate}
	\item Realios mašinos techninė įranga:
	\begin{enumerate}
		\item Procesorius - \enquote{\textbf{Intel Core i5-9600K}}.
		\item Grafinė vaizdo plokštė - \enquote{\textbf{Nvidia GeForce RTX 2070 Super}}.
		\item Operatyvioji atmintis - \enquote{\textbf{HyperX Predator Black}} (\textbf{16GB}, 3200MHz, DDR4, CL16).
		\item Pastovioji atmintis - \enquote{\textbf{Samsung SSD 970 EVO Plus}} (\textbf{500GB}, M.2 PCIe x4, 3500/3200 MB/s).
	\end{enumerate}

	\item Realios mašinos programinė įranga:
	\begin{enumerate}
		\item Operacinė sistema - \enquote{\textbf{Windows 10 Home N}} (versija: \textbf{1909}).
		\item \enquote{\textbf{Anaconda}} paketų ir aplinkų valdymo sistema (versija: \textbf{2019.10}).
		\item \enquote{\textbf{Python}} programavimo kalba (versija: \textbf{3.7.4}).
		\item \enquote{\textbf{Jupyter Notebook}} atviro kodo programa skirta kintančio kodo, matematinių funkcijų, teksto bei duomenų vizualizavimui (versija: \textbf{6.0.1}).
	\end{enumerate}
\end{enumerate}

\subsubsubsection{Ekseperimentinė}

\subsubsection{}

\printbibliography[heading=bibintoc] 
\end{document}
