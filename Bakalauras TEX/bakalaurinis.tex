% !TeX spellcheck = lt
\documentclass{VUMIFPSbakalaurinis}
\usepackage{algorithmicx}
\usepackage{algorithm}
\usepackage{algpseudocode}
\usepackage{amsfonts}
\usepackage{amsmath}
\usepackage{bm}
\usepackage{caption}
\usepackage{color}
\usepackage{float}
\usepackage{graphicx}
\usepackage{listings}
\usepackage{subfig}
\usepackage{url}
\usepackage{wrapfig}
\usepackage[table,xcdraw]{xcolor}
\usepackage[backend=biber]{biblatex}
\usepackage{enumitem}\setlist{nosep}
\usepackage{csquotes}
\usepackage{setspace}

%\onehalfspacing

% Titulinio aprašas
\university{Vilniaus universitetas}
\faculty{Informatikos institutas}
\department{Programų sistemos}
%\papertype{Bakalauro darbas}
\papertype{Bakalauro baigiamasis darbas}
%sustiprinto mokymosi algoritmu skirtu Sokoban zaidimo agento valdymui palyginimas
\title{Skatinamojo mokymosi taikymas žaidimo agento valdymo programos kūrimui}
\titleineng{Application of reinforcement learning to the software development for game agent management}
\author{Jokūbas Rusakevičius}
\supervisor{vyresn. m.d. Virginijus Marcinkevičius}
\reviewer{j. asist. Linas Petkevičius}
\date{Vilnius – \the\year}

\setmainfont{Palemonas}
\bibliography{bibliografija}

\begin{document}
\maketitle

\setcounter{page}{2}

\sectionnonumnocontent{Santrauka}
TODO: Santrauka
% Nurodomi iki 5 svarbiausių temos raktinių žodžių (terminų).
% Vienas terminas gali susidėti iš kelių žodžių.
\raktiniaizodziai{Skatinamasis mokymas, Sokoban žaidimas, aktorius-kritikas based metodai, raktinis žodis 4, raktinis žodis 5}   

\sectionnonumnocontent{Summary}
TODO: summary
\keywords{Reinforcement learning, Sokoban game, actor-critic based methods, keyword 4, keyword 5}

\tableofcontents

\sectionnonum{Įvadas}

\section{Teorija}


\subsection{Skatinamasis mokymasis}

\subsubsection{Markovo procesas}
\subsubsection{Sustiprinto mokymosi strategijos ieškojimas}
\subsubsection{Aktoriaus-kritiko principas}

\subsection{Gilusis mokymas}


\subsection{Konvoliuciniai neuroniniai tinklai}

\subsection{LSTM}


\section{Metodologija}
\subsection{Sokoban žaidimas}
\subsubsection{OpenAI Gym}

\subsection{Skatinamojo mokymosi bibliotekos parinkimas}
\subsubsection{Stable Baselines architektūra}
\subsubsubsection{A2C aprašymas}
\subsubsubsection{ACER aprašymas}
\subsubsubsection{POP2 aprašymas}

\section{Eksperimentai}
Šiame skyriuje aprašomi bakalauro darbo metu atlikti eksperimentai bei jiems paruošta eksperimentinė aplinka.

\subsection{Sokoban aplinkos paruošimas}
Šiame poskyryje yra aprašomas metodas eksperimento metu tiriamos Sokoban aplinkos paruošimui.
\subsubsection{}


\subsection{Eksperimentinė aplinka}
Eksperimentai atlikti naudojantis realia mašina su \enquote{Ubuntu} OS. Minėtoje mašinoje įdiegta \enquote{Anaconda} paketų valdymo ir dislokavimo sistema, naudojama aplinkų atskyrimui. Didžioji programinė dalis eksperimento atliekama \enquote{Jupyter Notebook} programavimo aplinkoje naudojantis \enquote{Python} kalba.

\subsubsection{Eksperimentinės aplinkos specifikacijos}
Eksperimentas atliekamas naudojantis realią \enquote{Ubuntu} mašiną.

\begin{enumerate}
	\item Kompiuterio techninė specifikacija:
	\begin{enumerate}
		\item Procesorius - \enquote{\textbf{Intel Core i5-9600K}} (6 branduoliai, bazinis greitis 3.70 GHz).
		\item Grafinė vaizdo plokštė - \enquote{\textbf{Nvidia GeForce RTX 2070 Super}}.
		\item Operatyvioji atmintis - \enquote{\textbf{HyperX Predator Black}} (\textbf{32GB}, 3200MHz, DDR4, CL16).
		\item Pastovioji atmintis - \enquote{\textbf{Western Digital}} (\textbf{1TB}).
	\end{enumerate}

	\item Kompiuterio programinė įranga:
	\begin{enumerate}
		\item Operacinė sistema - \enquote{\textbf{Ubuntu 18.04 LTS}} (versija: \textbf{18.04.4 LTS}).
		\item Paketų ir aplinkų valdymo sistema - \enquote{\textbf{Anaconda}}  (versija: \textbf{2020.02}).
		\item Programavimo kalba - \enquote{\textbf{Python}} (versija: \textbf{3.7.6}).
		\item Atviro kodo programa kintančio kodo, matematinių funkcijų, teksto bei duomenų vizualizavimui - \enquote{\textbf{Jupyter Notebook}}  (versija: \textbf{6.0.3}).
		\item ... \enquote{\textbf{Tensorflow}} (versija: \textbf{1.14.0}).
		\item ... \enquote{\textbf{OpenAI Gym}} (versija: \textbf{0.17.1}).
		\item ... \enquote{\textbf{Stable Baselines}} (versija: \textbf{2.10.1a0}).
		\item Išskirstyta VSC sistema pakeitimų sekimui kode - \enquote{\textbf{Git}} (versija: \textbf{2.23.0}) 
	\end{enumerate}
\end{enumerate}

\subsubsection{Ekseperimentinės aplinkos paruošimas}

\subsection{Eksperimento planas}
Darbo metu atliktas eksperimentas susideda iš trijų dalių. Šiame skyriuje yra aprašomi šių trijų eksperimentų planai: kaip bus atliekamas eksperimentas, kokia bus naudojama aplinka, kokių rezultatų yra tikimasi ir pan.
\subsubsubsection{Pirmo eksperimento planas: Geriausios strategijos ieškojimas}


\subsubsection{Eksperimentas}

\printbibliography[heading=bibintoc] 

\sectionnonum{Santrumpos}
Darbe naudojamų santrumpų paaiškinimai:
\begin{itemize}[label={}, leftmargin=*] %apgalvoti kaip padaryti šį sąrašą
	\item \textbf{RL} - (\textit{angl. Reinforcement Learning}) skatinamasis mokymas.
	\item \textbf{MDP} - (\textit{angl. Markov Decision Processes}) Markovo procesas.
	\item \textbf{VSC} - (\textit{angl. Version-Control System}) versijų tvarkymo sistema.
\end{itemize}

\appendix

\end{document}
