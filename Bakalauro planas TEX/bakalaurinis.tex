\documentclass{VUMIFPSbakalaurinis}
\usepackage{algorithmicx}
\usepackage{algorithm}
\usepackage{algpseudocode}
\usepackage{amsfonts}
\usepackage{amsmath}
\usepackage{bm}
\usepackage{caption}
\usepackage{color}
\usepackage{float}
\usepackage{graphicx}
\usepackage{listings}
\usepackage{subfig}
\usepackage{url}
\usepackage{wrapfig}
\usepackage[table,xcdraw]{xcolor}
\usepackage[backend=biber]{biblatex}
\usepackage{enumitem}\setlist{nosep}
\usepackage{csquotes}

% Titulinio aprašas
\university{Vilniaus universitetas}
\faculty{Informatikos institutas}
\department{Programų sistemos}
%\papertype{Bakalauro darbas}
\papertype{Bakalauro baigiamojo darbo planas}
\title{Sustiprinto mokymosi taikymas žaidimo agento valdymo programos kūrimui}
\titleineng{Application of reinforcement learning to the software development for game agent management}
\author{Jokūbas Rusakevičius}
\supervisor{vyresn. m.d. Virginijus Marcinkevičius}
\reviewer{}
\date{Vilnius – \the\year}

% Nustatymai
\setmainfont[ItalicFont 	= Palem3.2-it.ttf,
			BoldItalicFont	= Palem3.2-bi.ttf,
			BoldFont		= Palem3.2-bd.ttf]
			{Palem3.2-nm.ttf}
\bibliography{bibliografija}

\begin{document}
\maketitle

\setcounter{page}{2}


\subsectionnonum{Tyrimo objektas ir aktualumas}
Kompiuterių pajėgumui ir atliekamų operacijų per sekundę skaičiui nuolatos didėjant -- didėja ir lūkesčiai bei sprendžiamų uždavinių sudėtingumas. Dar reliatyviai neseniai sudėtingiausios programos ir kompiuterių sprendžiami uždaviniai susidėjo iš skaičiuotuvo operacijų ar žinučių perdavimo. Tačiau technologijoms tobulėjant, kiekvienam žmogui kišenėje besinešiojant pirmųjų kompiuterių kaip \enquote{ENIAC} \cite{computer_history} dydį pajuokiančius kompiuterinius įrenginius, natūraliai didėja ir jiems keliami iššūkiai. \par

Šiais laikais kompiuteriai gali simuliuoti atominius sprogimus, nuspėti orus ir atlikti kitas didžiulių skaičiavimo išteklių reikalaujančias užduotis \cite{supercomputers}. Tačiau užduoties sudėtingumą gali lemti ne tik milžiniškų išteklių skaičiaus reikalavimas. 2016 metais matėme, kaip \enquote{Google’s AlphaGo} nugalėjo pasaulio aukščiausio lygio \enquote{Go} žaidėją ir čempioną Ke Jie \cite{go}. Autonominiai gatvėmis važinėjantys automobiliai neišvengiamai artėja, o \enquote{Boston Dynamics} robotai stebina savo galimybėmis \cite{bostondynamics}. \par

Šie uždaviniai nėra trivialiai aprašomi ar išsprendžiami, jiems gali net neegzistuoti sprendimas. Tokiems uždaviniams spręsti yra naudojami mašininio mokymosi metodai (pvz. neuroniniai tinklai). Viena šių metodų šaka yra \enquote{sustiprintas mokymas} -- agento atliekami veiksmai yra reguliariai vertinami ir atitinkamai agentas yra apdovanojamas arba baudžiamas. \par

Sustiprintas mokymasis ir žaidimo agento valdymo programa yra pagrindiniai šio bakalauro darbo tiriamieji objektai.



\subsectionnonum{Darbo tikslas, keliami uždavyniai ir laukiami rezultatai}
Šio darbo \textbf{tikslas} -- išanalizavus populiariausius sustiprinto mokymosi algoritmus, pritaikyti labiausiai tinkamą algoritmą, pasirinkto žaidimo agentui valdyti, siekiant pagerinti agento žinomus veikimo rezultatus.\par

Darbui iškelti \textbf{uždaviniai}:\par

\begin{enumerate}
	\item Paruošti eksperimentinę aplinką ir agentą.
	\item Parinkti sustiprinto mokymosi algoritmą tinkamą agentui ir jį apmokyti.
	\item Palyginti agento veikimo efektyvumo priklausomybes nuo algoritmo pasirinkimo ir nuo algoritmo hiper parametrų.
	\item Pateikti rekomendacijas agento apmokymui parinktoje aplinkoje.
\end{enumerate}

Darbo metu laukiami \textbf{rezultatai}:

\begin{enumerate}
	\item Paruošta eksperimentinė aplinka ir agentas.
	\item Parinktas agento mokymui tinkamas sustiprinto mokymosi algoritmas.
	\item Palyginta agento veikimo efektyvumo priklausomybė nuo algoritmo pasirinkimo ir nuo algoritmo hiper parametrų.
	\item Pateiktos rekomendacijos agento apmokymui parinktoje aplinkoje.
\end{enumerate}

\subsectionnonum{Tyrimo metodas} 
Darbui atlikti bus naudojami šie tyrimo metodai:

\begin{enumerate}
	\item \textbf{Mokslinės literatūros analizė}.
	\item \textbf{Eksperimentas}.
	\subitem \textbf{Kiekybinis metodas} -- apmokyti pirminius agentus, stebėti jų elgseną. Eksperimentas atliekamas, vertinant agento veikimą keliais skirtingais kriterijais ir ieškoma tinkamiausio bei daugiausiai žadančio agento mokymo algoritmo tolimesniems eksperimentams.
	\subitem \textbf{Kokybinis metodas} -- su pasirinktais pradiniais agentais atliekamos gilesnės agento mokymosi parametrų manipuliacijos bei stebimi gaunamų rezultatų pakitimai.
	\item \textbf{Gautų duomenų ir rezultatų analizė}.
\end{enumerate}

\subsectionnonum{Numatomas darbo atlikimo procesas}
\textbf{Darbo procesas} susidės iš šių dalių:

\begin{enumerate}
	\item Visų pirma, eksperimentui atlikti bus paruošta eksperimentinė aplinka.
	\item Atliekama literatūros analizė ir parenkamas agento valdymo algoritmas.
	\item Atliekamas pirminis agento apmokymas, skirtas eksperimento bazei gauti bei įsitikinto aplinkos ir agento tarpusavio veikla. 
	\item Pradėjus pagrindinę eksperimento dalį bus ieškoma ir eksperimentuojama su metodais, kuriuos pritaikius būtų gaunami geresni nei aukščiau gaunami rezultatai.
	\item Galutinė visų atliktų tyrimų ir gautų rezultatų analizė.
	\item Pateikiami apibendrinimai ir rekomendacijos, kaip pagerinti agento apmokymą parinktai aplinkai.
\end{enumerate}


\subsectionnonum{Darbui aktualūs šaltiniai}

\begin{enumerate}
	\item \textbf{\cite{gym}} -- viešai prieinama \enquote{OpenAI} \cite{openai} aplinka realizuojanti tradicinį Japonišką sandėlių prižiūrėjimo video žaidimą \enquote{Sokoban}. Tai yra parinkta aplinka bakalauro darbo eksperimento daliai.
	\item \textbf{\cite{handson}} -- knyga -- mašininio ir neuroninio mokymosi algoritmų rinkinys. Rašoma apie daugelį populiariausių ir plačiausia žinomų skirtingų mokymosi algoritmų bei principų. Pateikiami pritaikymo bei panaudojimo pavyzdžiai.
	\item \textbf{\cite{algorithms}} -- klasikinė knyga apie sustiprintą mokymą. Šiame šaltinyje rašoma apie tai kas yra sustiprintas mokymas, kokie algoritmai egzistavo šaltinio leidimo metu bei kaip ir kada juos naudoti.
	\item \textbf{\cite{udemy1}} -- \enquote{Udemy} \cite{udemy} kursas apie dirbtinį intelektą bei pasiruošimą darbui su giliuoju sustiprintu mokymu. Kursas apžvelgia sustiprinto mokymosi algoritmų pritaikymo, programavimo ir dizaino principus.
	\item \textbf{\cite{udemy2}} - \enquote{Udemy} \cite{udemy} kursas apie dirbtinio intelekto pritaikymo naudojantis giliuoju mokymusi ir neuroniniais tinklais įvaldymą.	
\end{enumerate}

\printbibliography[heading=bibintoc] 
\end{document}
